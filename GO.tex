\documentclass[%
    ngerman,
   %draft,                  % Entwurf draft bedeutet draft=true, wird aber auch von graphicx verstanden
    final,
    twoside=true,           % zweiseitig
    DIV=14,                 % Satzspiegelberechnugn
    BCOR=12mm,              % Bindekorrektur
    fontsize=12pt,          % Schriftgröße
    paper=a4,               % Papierformat
    %toc=listof,             % Listen in das Inhaltsverzeichnis
    %toc=bib,                % Literatuverzeichnis ins Inhaltsverzeichnis
    headsepline=false,       %keine Trennlinie zwischen Kopf und Text
    footsepline=false,      % keine Linie im Dokumentfuss
    footnotes=multiple,     % Fußnoten werden getrennt
    abstract=false         % Zusammenfassung mit Wort "Zusammenfassung" wird selbstübernommen!
]{scrreprt}

%Design and Language Definition
    \usepackage{ngerman}
    \usepackage[T1]{fontenc}
    \usepackage[utf8]{inputenc}
    \usepackage[automark]{scrpage2} %Header definiton for Koma Script Classes

% Graphic Input and Styling
    \usepackage{xspace} %fügt hinter Makros einen entsprechenden Space ein wenn nötig oder eben nciht \xspace


%Special Stylings

    \usepackage{url}

%Pakete für die Quellcode Listings

%Typographisch interessante Pakete
    \usepackage{mathpazo}   %Palatino
    \linespread{1.05}       %Palatino braucht einen höheren Durchschuss
    \renewcommand{\sfdefault}{uop} %Optima clone classico als Überschrift
    %  more information on http://mactex-wiki.tug.org/wiki/index.php/Typefaces#Classico
    % install using the following command on Ubuntu (worked on 11.10):
    %     sudo su      (or)      su
    %     apt-get install texlive-base-bin
    %     getnonfreefonts-sys classico
    %     echo -e "\nMap uop.map\n" >> /var/lib/texmf/web2c/updmap.cfg
    %     updmap-sys
    %     exit
    \usepackage[tracking=true]{microtype} % Randkorrektur und andere Anpassungen
    \DeclareMicrotypeSet*[tracking]{my}   % Sperrt Kapitälchen
      { font = */*/*/sc/* }%
    \SetTracking{ encoding = *, shape = sc }{ 45 }%
    \KOMAoptions{DIV=last} % Satzspiegel neu berechnen, da Paladino als Schrift gewählt


%References to Internet and within the document !!!always last package!!!
    \usepackage[
        pdftex,
    %   % Farben fuer die Links
       colorlinks=true,         % Links erhalten Farben statt Kaeten
       urlcolor=cyan,    % \href{...}{...} external (URL)
       filecolor=pdffilecolor,  % \href{...} local file
       linkcolor=red,  %\ref{...} and \pageref{...}
       citecolor=green,  %
    %   % Links
    %      raiselinks=true,             % calculate real height of the link
    %   breaklinks,              % Links berstehen Zeilenumbruch
    %   backref=page,            % Backlinks im Literaturverzeichnis (section, slide, page, none)
    %   pagebackref=true,        % Backlinks im Literaturverzeichnis mit Seitenangabe
    %   verbose,
    %   hyperindex=true,         % backlinkex index
    %      linktocpage=true,        % Inhaltsverzeichnis verlinkt Seiten
    %   hyperfootnotes=false,     % Keine Links auf Fussnoten
    %   % Bookmarks
       bookmarks=true,          % Erzeugung von Bookmarks fuer PDF-Viewer
       bookmarksopenlevel=-1,    % Gliederungstiefe der Bookmarks
    %   bookmarksopen=true,      % Expandierte Untermenues in Bookmarks
    %   bookmarksnumbered=true,  % Nummerierung der Bookmarks
       bookmarkstype=toc,       % Art der Verzeichnisses
    %   % Anchors
    %   plainpages=false,        % Anchors even on plain pages ?
    %   pageanchor=true,         % Pages are linkable
    %   % PDF Informationen
       pdftitle={Geschäftsordnung für Plenen der ZaPF},             % Titel
       pdfauthor={AK GO der ZaPF},                             % Autor
       pdfcreator={Zusammenkunft aller Physik-Fachschaften},        % Ersteller
       pdfproducer={StAPF},   %Produzent
       pdftoolbar=true,         % Shows PDFToolbar
       pdfdisplaydoctitle=true, % Dokumententitel statt Dateiname im Fenstertitel
       pdfstartview=FitV,       % Dokument wird Fit Vertical geoeffnet
       pdfpagemode=UseOutlines, % Bookmarks im Viewer anzeigen
    %   pdfpagelabels=true,           % set PDF page labels
       pdfpagelayout=TwoPageRight%, % zweiseitige Darstellung: ungerade Seiten
    %                                        % rechts im PDF-Viewer
    %   %pdfpagelayout=SinglePage, % einseitige Darstellung
    ]{hyperref}




%Special Commands

\renewcommand \thesection {\Roman{section}}
\setlength{\parindent}{0pt}


\begin{document}

\chapter*{Geschäftsordnung\footnote{beschlossen 2005 im Abschlussplenum der Sommer ZaPF in Erlangen,\\ geändert 2007 auf der Sommer ZaPF in Berlin\\geändert 2008 auf der Sommer ZaPF in Konstanz\\geändert 2008 auf der Winter ZaPF in Aachen } für Plenen der ZaPF}


\noindent \textbf{Die männliche Anrede gilt im folgenden sowohl für
weibliche als auch für männliche
 TeilnehmerInnen der ZaPF.}

\section{Ablauf eines Plenums:}
\begin{enumerate}

\item{Sitzungen der ZaPF sind öffentlich.}

\item{Die Sitzungsleitung wird von der die ZaPF organisierenden Fachschaft vorgeschlagen und im
Plenum abgestimmt.}

\item{Zu Beginn der Sitzung wird ein Protokollführer gewählt, das Protokoll der Sitzung wird im ZaPFReader
abgedruckt.}

\item{Die Beschlussfähigkeit ist festzustellen (Details siehe Abschnitt Beschlüsse).}

\item{Anschließend wird die Tagesordnung bekanntgegeben und abgestimmt. Diese Tagesordnung ist
bindend. Im Abschlussplenum sollte es immer einen Tagesordnungspunkt „Berichte der AK“ geben.
Sollte ein AK eine Abstimmung wünschen, ist dies als Antrag einzureichen.}

\item{ Ist in einer Sitzung strittig, wie eine Bestimmung dieser Geschäftsordnung auszulegen oder wie eine
Lücke zu schließen ist, so kann die Auslegungsfrage mit Wirkung für die gesamte Sitzung durch die
Sitzungsleitung entschieden werden.}

\item{Die Sitzungsleitung kann die Sitzung unterbrechen, dies sollte in der Regel jedoch 10 Minuten nicht
überschreiten.}
\end{enumerate}

\noindent
\section{Anträge:}
\begin{enumerate}
\item{Anträge (z.B. für Tagesordnungspunkte oder Abstimmungen) sind mindestens eine Stunde vor
Beginn des Plenums schriftlich bei der die ZaPF ausrichtenden Fachschaft einzureichen. Dies gilt
insbesondere für Texte, über die abgestimmt werden soll. Die AK haben dafür zu sorgen, dass dies
rechtzeitig geschieht. Der Antragssteller muss im Plenum anwesend sein.}

\item{Anträge, die nach dieser Frist eingereicht werden, sind Initiativanträge und müssen von mindestens
zwei Personen aus verschiedenen Fachschaften getragen werden. Auch diese Anträge sind der
Sitzungsleitung schriftlich mitzuteilen.}

\item{Änderungen dieser Geschäftsordnung sind nicht durch Initiativanträge möglich. Sie müssen zur
Abstimmung im Anfangsplenum mindestens 7 Tage vor dem Anfangsplenum
der ZaPF bekanntgemacht werden, z.B. über die Mailingliste. Zur
Abstimmung im Abschlussplenum müssen Änderungsanträge der
Geschäftsordnung mindestens einen Tag vor dem Plenum bekanntgegeben
werden.}

\item{Geschäftsordnungsanträge (GO-Anträge, werden durch das Heben beider Arme signalisiert) sind spätestens
vor der nächsten Wortmeldung bzw. Abstimmung zu behandeln und
abzustimmen. Es ist nur eine Für- und eine Gegenrede erlaubt. Eine
inhaltliche Gegenrede ist einer formellen vorzuziehen. Eine
Diskussion findet nicht statt. In der Abstimmung ist (bis auf unten
angegebene Ausnahmen) eine einfache Mehrheit erforderlich. Gibt es
keine Gegenrede gilt der Antrag als angenommen.}


\begin{quote}
\textbf{Geschäftsordnungsanträge sind insbesondere Anträge:}
\begin{itemize}
\item{zur Änderung der Tagesordnung}
\item{zur erneuten Feststellung der Beschlussfähigkeit (ohne Abstimmung, ohne Gegenrede)}
\item{zur Unterbrechung der Sitzung}
\item{zur Vertagung eines Verhandlungsgegenstandes}
\item{zur Begrenzung der Redezeit}
\item{zum Schluss der Rednerliste (nach Annahme des Antrages können sich noch Redner auf die Liste
setzen lassen, anschließend wird die Liste geschlossen, weitere Wortmeldungen sind dann nicht
mehr möglich)}
\item{Wiedereröffnung der Redeliste *}
\item{geschlossene Sitzung (jeweils nur für einen Tagesordnungspunkt)}
\item{Zulassung Einzelner zur geschlossenen Sitzung}
\item{zum Schluss der Debatte (die Diskussion wird nach Annahme des Antrages sofort abgebrochen, eine
Abstimmung zum Thema wird ggf. sofort durchgeführt)*}
\item{zur Anzweiflung einer Abstimmung}
\item{zur Verweisung in eine Arbeitsgruppe}
\item{Nichtbefassung *}
\item{geheime Abstimung (ohne Gegenrede)}

\end{itemize}
Mit einem * gekennzeichnete Anträge erfordern eine 2/3-Mehrheit.
\end{quote}
\end{enumerate}
\textbf{Kommentar:} {\footnotesize\textit{Geschäftsordnungsanträge
sind dazu gedacht, zu verhindern, dass eine Diskussion sich ins
Absurde zieht. Sie sind mit äußerster Vorsicht anzuwenden. Bei der
Abstimmung über einen Geschäftsordnungsantrag sollte man vorher
dreimal darüber nachdenken, ob man ihm zustimmt, da Ende der Debatte
Ende der Debatte bedeutet. Geschäftsordnungsanträge werden durch
heben beider Arme bekanntgegeben. Geschäftsordnungsanträge können
als Mittel zu einer Schlammschlacht genutzt werden, jedoch sollte
bedacht werden, dass wir als Physiker sachliche Diskussionen führen
sollten und auch einsehen sollten, wenn die Mehrheit einen Antrag
nicht unterstützt. Abstimmungen ohne jegliche Gegenrede sollten nur
mit äußerster Vorsicht angenommen werden. Formale Gegenrede bedeutet
nur bekanntzugeben, dass man dagegen ist,
inhaltliche Gegenrede beinhaltet eine Begründung.}}\\[1ex]

\section{Beschlüsse und Wahlen}
\begin{enumerate}
\item{Die Beschlussfähigkeit ist gegeben, wenn 15\footnote{Dies entspricht nach unserem Kenntnisstand etwa 1/4 der Fachschaften Physik.} Physik Fachschaften im Plenum anwesend sind.}

\item{Ein Beschluss gilt als angenommen, wenn die Anzahl der Ja-Stimmen größer ist als die Summe aus
Enthaltungen und Nein-Stimmen. Sollte die Zahl der Enthaltungen die
Summe der Ja- und Nein- Stimmen überwiegen, wird die Abstimmung
einmalig wiederholt. Falls in der erneuten Abstimmung wiederum die
Zahl der Enthaltungen überwiegt, gilt der Antrag als abgelehnt. Die
Abstimmung geschieht durch deutliches Handheben, eine geheime
Abstimmung kann beantragt werden. Eine schriftliche Stimmabgabe ist
bei vorzeitiger Abreise möglich, es ist jedoch bei geheimer
Abstimmung auf Wahrung des Wahlgeheimnisses zu achten.
Stimmrechtsübertragung ist nicht möglich. Anträge zur Abstimmung
sind positiv zu formulieren.}

\item{Stimmberechtigt für Meinungsbilder ist jeder angemeldete Teilnehmer der
ZaPF.}

\item{Jede Fachschaft hat eine Stimme; wie sie abstimmt, ist innerhalb der jeweiligen
Fachschaft zu regeln. Den Fachschaften ist Zeit zur Beratung zu
gewähren. Eine geheime Abstimmung ist möglich.}

\item{Änderungsanträge ändern den Wortlaut eines Antrages, aber nicht das Wesen. Sie können von jedem
Teilnehmer gestellt werden. Änderungsanträge sind vor dem eigentlichen Antrag zu beschliessen.
Soweit das Plenum den Änderungsanträgen zustimmt oder sie vom Hauptantragsteller übernommen
werden, wird der Hauptantrag in der geänderten Fassung zur Beschlussfassung gestellt. Der
Antragsteller hat bis zur endgültigen Beschlussfassung das Recht, auch eine geänderte Fassung
seines Antrages zurückzuziehen.}

\item{Bei konkurrierenden Anträgen ist die Beschlussfassung wie folgt durchzuführen:
Geht ein Antrag weiter als ein anderer, so ist über den weitergehenden zuerst zu beschließen. Wird
dieser angenommen, so sind weniger weit gehende Anträge erledigt. Lässt sich ein Weitergehen
nicht feststellen, so bestimmt sich die Reihenfolge, in der die konkurrierenden Anträge zur
Beschlussfassung gestellt werden, aus der Reihenfolge der Antragsstellung. Lässt sich diese nicht
mehr feststellen, entscheidet die Sitzungsleitung.}

\item{Wahlen sind grundsätzlich geheim durchzuführen. Die Kandidaten stellen sich vor der Wahl kurz
dem Plenum vor. Dem Plenum ist die Möglichkeit zu geben, unter Ausschluss des Kandidaten zu
diskutieren.\\
JedeFachschaft besitzt eine Stimme. Ein Kandidat gilt als gewählt,
wenn er mehr Ja-Stimmen als Nein-Stimmen erhält und die Wahl
annimmt. Sollten mehr Kandidaten gewählt werden, als Posten zur
Verfügung stehen, werden sie nach Anzahl der Ja-Stimmen besetzt.}

\item{Abwahlen sind auch bei Abwesenheit des Betroffenen möglich und bedürfen einer 2/3-Mehrheit.
Der Betroffene ist jedoch nach Möglichkeit anzuhören.}
\end{enumerate}


\end{document}
