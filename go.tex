\documentclass[draft,12pt,oneside]{scrreprt}

% Sprache und Encodings
\usepackage[ngerman]{babel}
\usepackage[T1]{fontenc}
\usepackage[utf8]{inputenc}

% Typographisch interessante Pakete
\usepackage{microtype} % Randkorrektur und andere Anpassungen

% References to Internet and within the document !!!always last package!!!
\usepackage[pdftex,colorlinks=false,
pdftitle={Geschäftsordnung für Plenen der ZaPF},
pdfauthor={Zusammenkunft aller Physikfachschaften},
pdfcreator={pdflatex},
pdfdisplaydoctitle=true]{hyperref}

% Absaetze nicht Einruecken
\setlength{\parindent}{0pt}

\usepackage{draftwatermark}
\SetWatermarkText{Entwurf}
\SetWatermarkScale{5}

\begin{document}

\chapter*{Geschäftsordnung für Plenen der ZaPF}

\textbf{Die männliche Anrede gilt im folgenden sowohl für weibliche als auch für
männliche TeilnehmerInnen der ZaPF.}

\section{Ablauf eines Plenums}

\begin{enumerate}
  \item Sitzungen der ZaPF sind öffentlich.

  \item Die Sitzungsleitung wird von der die ZaPF organisierenden Fachschaft
        vorgeschlagen und im Plenum abgestimmt.

  \item Zu Beginn der Sitzung wird ein Protokollführer gewählt, das Protokoll
        der Sitzung wird im ZaPFReader abgedruckt.

  \item Die Beschlussfähigkeit ist festzustellen (Details siehe Abschnitt Beschlüsse).

  \item Anschließend wird die Tagesordnung bekanntgegeben und abgestimmt.
        Diese Tagesordnung ist bindend.
        Im Anfangsplenum werden nach Abstimmung der Tagesordnung die
        Vertrauenspersonen gewählt.
        Im Abschlussplenum sollte es immer einen Tagesordnungspunkt ``Berichte
        der AK'' geben.
        Sollte ein AK eine Abstimmung wünschen, ist dies als Antrag einzureichen.

  \item Ist in einer Sitzung strittig, wie eine Bestimmung dieser Geschäftsordnung
        auszulegen oder wie eine Lücke zu schließen ist, so kann die Auslegungsfrage
        mit Wirkung für die gesamte Sitzung durch die Sitzungsleitung entschieden
        werden.

  \item Die Sitzungsleitung kann die Sitzung unterbrechen, dies sollte in der
        Regel jedoch 10 Minuten nicht überschreiten.
\end{enumerate}

\section{Anträge}

\begin{enumerate}
  \item Anträge (z.B. für Tagesordnungspunkte oder Abstimmungen) sind mindestens
        eine Stunde vor Beginn des Plenums schriftlich bei der die ZaPF
        ausrichtenden Fachschaft einzureichen.
        Dies gilt insbesondere für Texte, über die abgestimmt werden soll.
        Die AK haben dafür zu sorgen, dass dies rechtzeitig geschieht.
        Der Antragssteller muss im Plenum anwesend sein oder kann einen Vertreter
        benennen und muss dies der Redeleitung mitteilen.
        Der Vertreter ist dann der neue Antragsteller.

  \item Anträge, die nach dieser Frist eingereicht werden, sind Initiativanträge
        und müssen von mindestens zwei Personen aus verschiedenen Fachschaften
        getragen werden.
        Auch diese Anträge sind der Sitzungsleitung schriftlich mitzuteilen.

  \item Änderungen dieser Geschäftsordnung sind nicht durch Initiativanträge möglich.
        Sie müssen zur Abstimmung im Anfangsplenum mindestens 7 Tage vor dem
        Anfangsplenum der ZaPF bekanntgemacht werden, z.B. über die Mailingliste.
        Zur Abstimmung im Abschlussplenum müssen Änderungsanträge der Geschäftsordnung
        spätestens 15:00 Uhr am Tag vor dem Abschlussplenum bekanntgegeben werden.
        Die Änderung der Geschäftsordnung bedarf einer absoluten Mehrheit.
        Die Geschäftsordnungsanträge, die einer 2/3-Mehrheit bedürfen, können nur
        explizit und mit einer 2/3-Mehrheit geändert werden.

  \item Geschäftsordnungsanträge (GO-Anträge, werden durch das Heben beider Arme
        signalisiert) sind spätestens vor der nächsten Wortmeldung bzw. Abstimmung
        zu behandeln und abzustimmen.
        Es ist nur eine Für- und eine Gegenrede erlaubt.
        Eine inhaltliche Gegenrede ist einer formellen vorzuziehen.
        Eine Diskussion findet nicht statt.
        In der Abstimmung ist (bis auf unten angegebene Ausnahmen) eine einfache
        Mehrheit erforderlich.
        Gibt es keine Gegenrede gilt der Antrag als angenommen.

    \textbf{Geschäftsordnungsanträge sind insbesondere Anträge:}
    \begin{itemize}
      \item zur Änderung der Tagesordnung

      \item zur erneuten Feststellung der Beschlussfähigkeit
            (ohne Abstimmung, ohne Gegenrede)

      \item zur Unterbrechung der Sitzung

      \item zur Vertagung eines Verhandlungsgegenstandes in einen anderen Tagesordnungspunkt

      \item zur Begrenzung der Redezeit

      \item zum Schluss der Rednerliste (nach Annahme des Antrages können sich
            noch Redner auf die Liste setzen lassen, anschließend wird die Liste
            geschlossen, weitere Wortmeldungen sind dann nicht mehr möglich)

      \item Wiedereröffnung der Redeliste *

      \item geschlossene Sitzung (jeweils nur für einen Tagesordnungspunkt)

      \item Zulassung Einzelner zur geschlossenen Sitzung

      \item zum Schluss der Debatte (die Diskussion wird nach Annahme des
            Antrages sofort abgebrochen, eine Abstimmung zum Thema wird ggf.
            sofort durchgeführt)*

      \item zur Anzweiflung einer Abstimmung (ohne Gegenrede, ohne Abstimmung)

      \item zur Schließung der Redeliste und Verweisung in eine Arbeitsgruppe mit
            Recht auf ein Meinungsbild im Plenum *

      \item Nichtbefassung *

      \item geheime Abstimung (ohne Gegenrede, ohne Abstimmung)

      \item Neuwahl der Redeleitung unter Benennung eines Gegenkandidaten

      \item Neuwahl des Protokollanten unter Benennung eines Gegenkandidaten

      \item Einholung eines Meinungsbildes im Plenum
    \end{itemize}
    Mit einem * gekennzeichnete Anträge erfordern eine 2/3-Mehrheit.
\end{enumerate}

\textbf{Kommentar:} \textit{\footnotesize Geschäftsordnungsanträge sind dazu
  gedacht, zu verhindern, dass eine Diskussion sich ins Absurde zieht.
  Sie sind mit äußerster Vorsicht anzuwenden.
  Bei der Abstimmung über einen Geschäftsordnungsantrag sollte man vorher dreimal
  darüber nachdenken, ob man ihm zustimmt, da Ende der Debatte Ende der Debatte
  bedeutet.
  Geschäftsordnungsanträge werden durch heben beider Arme bekanntgegeben.
  Geschäftsordnungsanträge können als Mittel zu einer Schlammschlacht genutzt
  werden, jedoch sollte bedacht werden, dass wir als Physiker sachliche
  Diskussionen führen sollten und auch einsehen sollten, wenn die Mehrheit einen
  Antrag nicht unterstützt.
  Abstimmungen ohne jegliche Gegenrede sollten nur mit äußerster Vorsicht
  angenommen werden.
  Formale Gegenrede bedeutet nur bekanntzugeben, dass man dagegen ist, inhaltliche
  Gegenrede beinhaltet eine Begründung.}

\section{Beschlüsse und Wahlen}

\begin{enumerate}
  \item Die Beschlussfähigkeit ist gegeben, wenn 15\footnotemark Physik Fachschaften
        im Plenum anwesend sind.
        \footnotetext{Dies entspricht nach unserem Kenntnisstand etwa 1/4 der
          Fachschaften Physik.}

  \item Ein Beschluss gilt als angenommen, wenn die Anzahl der Ja-Stimmen größer
        ist als die Summe aus Enthaltungen und Nein-Stimmen.
        Sollte die Zahl der Enthaltungen die Summe der Ja- und Nein- Stimmen
        überwiegen, wird die Abstimmung einmalig wiederholt.
        Falls in der erneuten Abstimmung wiederum die Zahl der Enthaltungen
        überwiegt, gilt der Antrag als abgelehnt.
        Die Abstimmung geschieht durch deutliches Handheben, eine geheime
        Abstimmung kann beantragt werden.
        Eine schriftliche Stimmabgabe ist bei vorzeitiger Abreise möglich, es ist
        jedoch bei geheimer Abstimmung auf Wahrung des Wahlgeheimnisses zu achten.
        Stimmrechtsübertragung ist nicht möglich.
        Anträge zur Abstimmung sind positiv zu formulieren.

  \item Stimmberechtigt für Meinungsbilder ist jeder angemeldete Teilnehmer der
        ZaPF.

  \item Jede Fachschaft hat eine Stimme; wie sie abstimmt, ist innerhalb der
        jeweiligen Fachschaft zu regeln.
        Den Fachschaften ist Zeit zur Beratung zu gewähren.
        Eine geheime Abstimmung ist möglich.

  \item Änderungsanträge ändern den Wortlaut eines Antrages, aber nicht das Wesen.
        Sie können von jedem Teilnehmer gestellt werden.
        Änderungsanträge sind vor dem eigentlichen Antrag zu beschliessen.
        Soweit das Plenum den Änderungsanträgen zustimmt oder sie vom
        Hauptantragsteller übernommen werden, wird der Hauptantrag in der
        geänderten Fassung zur Beschlussfassung gestellt.
        Der Antragsteller hat bis zur endgültigen Beschlussfassung das Recht,
        auch eine geänderte Fassung seines Antrages zurückzuziehen.

  \item Bei konkurrierenden Anträgen ist die Beschlussfassung wie folgt durchzuführen:
        Geht ein Antrag weiter als ein anderer, so ist über den weitergehenden
        zuerst zu beschließen.
        Wird dieser angenommen, so sind weniger weit gehende Anträge erledigt.
        Lässt sich ein Weitergehen nicht feststellen, so bestimmt sich die
        Reihenfolge, in der die konkurrierenden Anträge zur Beschlussfassung
        gestellt werden, aus der Reihenfolge der Antragsstellung.
        Lässt sich diese nicht mehr feststellen, entscheidet die Sitzungsleitung.

  \item Wahlen sind grundsätzlich geheim durchzuführen.
        Die Kandidaten stellen sich vor der Wahl kurz dem Plenum vor.
        Dem Plenum ist die Möglichkeit zu geben, unter Ausschluss des Kandidaten
        zu diskutieren.
        Diese Diskussion wird nicht protokolliert.
        Jede Fachschaft besitzt eine Stimme.
        Ein Kandidat gilt als gewählt, wenn er mehr Ja-Stimmen als Nein-Stimmen
        und mindestens acht\footnotemark Ja-Stimmen erhält und die Wahl annimmt.
        Sollten mehr Kandidaten gewählt werden, als Posten zur Verfügung stehen,
        werden sie nach Anzahl der Ja-Stimmen besetzt.
        Die Wahl der Vertrauenspersonen wird in einem eigenen Absatz geregelt.
        \footnotetext{Das Minimum von acht Ja-Stimmen bewirkt, dass ein Kandidat
          mindestens die absolute Mehrheit der zur Beschlussfähigkeit notwendigen
          Stimmen erhalten muss.
          Enthaltungen sind möglich und wirken wie nicht oder ungültig abgegebene
          Stimmen.}


  \item Im Anfangsplenumg werden sechs Vertrauenspersonen gewählt.
        Die Wahl der Vertrauenspersonen\footnotemark erfolgt per Wahl durch
        Zustimmung aus einem Pool von Teilnehmern der ZaPF.
        Bewerbungen hierfür müssen bis spätestens zu Beginn des Anfangsplenums
        in schriftlicher Form an eine, bis spätestens zwei Wochen vor Beginn der
        ZaPF durch die ausführende Fachschaft bekanntzugebende, Adresse erfolgen.

        Der so bestimmten Gruppe muss anschließend mit absoluter Mehrheit vom
        Plenum das Vertrauen ausgesprochen werden, damit sie als gewählt gelten.
        Sind die ersten sechs Personen genannter Gruppe vom gleichen Geschlecht,
        ersetzt die Person eines anderen Geschlechts mit den meisten Stimmen die
        sechste Person in der Rangfolge.
        Sollten sich nur Personen eines Geschlechts beworben haben, ist diese
        Regelung irrelevant.

        Bei weniger als sieben sich bewerbenden Menschen muss der kompletten Gruppe
        das Vertrauen mit absoluter Mehrheit vom Plenum ausgesprochen werden,
        damit sie als gewählt gelten.
        Die Wahl durch Zustimmung entfällt hierbei.

        Stimmberechtigt ist jeder angemeldete Teilnehmer der ZaPF.
        Die Wahlen sind geheim.
        Eine Personaldebatte findet nicht statt, die Kandiaten dürfens sich
        jedoch dem Plenum vorstellen.
        Die Stimmverteilung wird nicht bekanntgegeben.
        Die gewählten Vertrauenspersonen werden in alphabetischer Reihenfolge
        vom Wahlausschuss veröffentlicht.

        Darüber hinaus nominiert die austragende Fachschaft zwei Vertrauenspersonen
        aus ihrer Fachschaft, diese müssen nicht vom Plenum bestätigt werden.

        Die gewählten Vertrauenspersonen nehmen ihre Aufgaben bis zum Beginn der
        nächsten ZaPF wahr.

        \footnotetext{\url{https://vmp.ethz.ch/zapfwiki/index.php/WiSe13_Beschl\%C3\%BCsse\#Vertrauenspersonen_.28Selbstverpflichtung.29}}

  \item Abwahlen sind auch bei Abwesenheit des Betroffenen möglich und bedürfen
        einer 2/3-Mehrheit.
        Der Betroffene ist jedoch nach Möglichkeit anzuhören.
\end{enumerate}

\section*{Versionshistorie}

Diese Geschäftsordnung wurde auf dem Abschlussplenum der Sommer-ZaPF 2005 in
Erlangen beschlosse.
Inhaltliche Änderungen wurden vorgenommen auf der:

\begin{itemize}

  \item Sommer-ZaPF 2007 in Berlin

  \item Sommer-ZaPF 2008 in Konstanz,

  \item Winter-ZaPF 2008 in Aachen,

  \item Sommer-ZaPF 2009 in Göttingen,

  \item und auf der Winter-ZaPF 2010 in Frankfurt.

\end{itemize}


\end{document}
